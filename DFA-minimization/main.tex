\documentclass{beamer}
\usetheme{metropolis}
\usecolortheme{seahorse}

\usepackage{kotex, csquotes}

\usepackage{fontspec}
% \setsansfont{Noto Sans}
% \setsanshangulfont{Noto Sans CJK KR}
% \setmonofont{Iosevka Fixed Slab}
% 
\usepackage{unicode-math}
\setmathfont{STIX Two Math}

\title{DFA minimization}
\subtitle{Hopcroft's algorithm}
\author{문순원}
\date{}

\begin{document}

\begin{frame}
  \titlepage
\end{frame}

\begin{frame}{Lexer generator's work}
  \begin{block}{From regex to DFA}
    \begin{itemize}
      \item convert regex into \(\varepsilon\)-NFA (Thompson's construction)
      \item convert \(\varepsilon\)-NFA into NFA (Dealing with \(\varepsilon\)-closure)
      \item convert NFA into DFA (Powerset construction)
      \alert<2> {\item minimize DFA (Hopcroft's algorithm)}
    \end{itemize}
  \end{block}
\end{frame}

\begin{frame}{Definition of minimal DFA and some facts about it}
  \small
  DFA \(M\) is said to be \textit{minimal} if no DFA with fewer states recognizes \(L(M)\). \\[1em]
  The minimial DFA for any recognizable language is unique up to isomorphism. \\[1em]
  Given an arbitrary DFA \(M\), one can effectively construct the minimal DFA equivalent to \(M\).
\end{frame}

\begin{frame}[t]{State equivalence and reduced DFA}
  Let DFA \(M = (Q, \Sigma, \delta, q_0, F)\) \\[1em]

  Two states \(a\) and \(b\) are \textit{equivalent} if \( (a,b) \in \rho_M \) where
  \[ \rho_M = \{ (a,b) \mid (\forall w \in \Sigma^*) \ (\delta_w(a) \in F \Leftrightarrow \delta_w(b) \in F) \} \]

  \pause
  DFA \(M\) is \textit{reduced} if equivalence implies identity. i.e.
  \[ (a,b) \in \rho_M \Rightarrow a = b \]

  \pause
  { \color{blue}
    Reduced DFA is minimal! \\
    How can we construct reduced DFA?
  }
\end{frame}

\begin{frame}[t]{Congruence and quotient of DFA}
  \only<1-2> {
    An equivalence relation \(\rho \subseteq Q \times Q\) is a \textit{congruence} of \(M\) if
    \begin{itemize}
      \item \(\delta_x\) is compatible with \(\rho\) \\
            \( (a,b) \in \rho \Rightarrow (\delta_x(a), \delta_x(b)) \in \rho \)
      \item \(F\) is a union of some \(\rho\)-classes (\(\rho\) saturates \(F\)) \\
            \( [a]_\rho \in F/\rho \Rightarrow [a]_\rho \subseteq F \)
    \end{itemize}
  }
  \only<2> {
    The \textit{quotient DFA} \(M/\rho\) by a congruence \(\rho\) is defined as follows
    \[
      \begin{array}{c}
        M/\rho = (Q/\rho, \Sigma, \delta/\rho, [q_0]_{\rho}, F/\rho) \\
        \textit{where} \quad (\delta/\rho)_x([a]_{\rho}) = [\delta_x(a)]_{\rho}
      \end{array}
    \]
    
    Quotient preserves recognizable language. i.e. \( L(M) = L(M/\rho) \)
  }

  \only<3-4> {
    An equivalence relation \(\rho\) is coarser than \(\pi\) if
    \[ (a,b) \in \pi \Rightarrow (a,b) \in \rho \]
  }
  \only<4> {
    \\ Following two statements are true
    \begin{itemize}
      \item \(\rho_M\) is the coarsest congruence
      \item \(M/\rho_M\) is reduced. Hence it's minimal
    \end{itemize}
  }
\end{frame}

\begin{frame}{The classical minimization algorithm}
  We can find \(\rho_M\) by following iterative algorithm \\[1em]

  Let \(\rho_i\) be a sequence of equivalence relation on \(Q\) such that
  \[
    \begin{array}{ll}
      \rho_0 &= \{ (a,b) \mid a, b \in F \} \cup \{ (a,b) \mid a, b \in Q \setminus F \} \\
      \rho_{i+1} &= \{ (a,b) \in \rho_i \mid (\forall x \in \Sigma) \ (\delta_x(a), \delta_x(b)) \in \rho_i \}
    \end{array}
  \]
  
  \pause
  Then there exists \(k \in \mathbb{N} \) s.t. \(\rho_{k+1} = \rho_k\) and \\
  \(\rho_k\) is the coarsest congruence of \(M\). (i.e. \(\rho_M = \rho_k\))
\end{frame}

\begin{frame}{Hopcroft's algorithm}
  \footnotesize
  \[
    \begin{array}{l}
      Q/\rho \leftarrow \{ F, Q \setminus F \} \\
      L \leftarrow \{ F \} \\
      \text{while there exists} \ A \in L \ \text{do} \\
      \qquad L \leftarrow L \setminus \{A\} \\
      \qquad \text{for each} \ x \in \Sigma \ \text{do} \\
      \qquad \qquad \text{let} \ X = \delta_x^{-1}(A) \\
      \qquad \qquad \text{for each} \ Y \in Q/\rho \ \text{s.t.} \
          (Y' = Y \cap X \neq \emptyset) \wedge (Y'' = Y \setminus X \neq \emptyset) \ \text{do} \\
      \qquad \qquad \qquad Q/\rho \leftarrow (Q/\rho \setminus \{Y\}) \cup \{Y', Y''\} \\
      \qquad \qquad \qquad \text{if} \ Y \in L \ \text{then} \\
      \qquad \qquad \qquad \qquad L \leftarrow (L \setminus \{Y\}) \cup \{Y', Y''\} \\
      \qquad \qquad \qquad \text{else} \\
      \qquad \qquad \qquad \qquad L \leftarrow L \cup \{ \text{min}(Y', Y'') \} \\
      \qquad \qquad \text{end} \\
      \qquad \text{end} \\
      \text{end}
    \end{array}
  \]
  \pause
  \color{blue} \(L \subseteq Q/\rho\) is a loop invariant!
\end{frame}

\begin{frame}{Modified version}
  \footnotesize
  {\color{blue} Let \( R = Q/\rho \setminus L \)}
  \[
    \begin{array}{l}
      R \leftarrow \{ Q \setminus F \} \\
      L \leftarrow \{ F \} \\
      \text{while there exists} \ A \in L \ \text{do} \\
      \qquad L \leftarrow L \setminus \{A\} \\
      \qquad R \leftarrow R \cup \{A\} \\
      \qquad \text{for each} \ x \in \Sigma \ \text{do} \\
      \qquad \qquad \text{let} \ X = \delta_x^{-1}(A) \\
      \qquad \qquad \text{for each} \ Y \in R \ \text{s.t.} \
          (Y' = Y \cap X \neq \emptyset) \wedge (Y'' = Y \setminus X \neq \emptyset) \ \text{do} \\
      \qquad \qquad \qquad R \leftarrow (R \setminus \{Y\}) \cup \{\text{max}(Y', Y'')\} \\
      \qquad \qquad \qquad L \leftarrow L \cup \{\text{min}(Y', Y'')\} \\
      \qquad \qquad \text{end} \\
      \qquad \qquad \text{for each} \ Y \in L \ \text{s.t.} \
          (Y' = Y \cap X \neq \emptyset) \wedge (Y'' = Y \setminus X \neq \emptyset) \ \text{do} \\
      \qquad \qquad \qquad L \leftarrow (L \setminus \{Y\}) \cup \{Y', Y''\} \\
      \qquad \qquad \text{end} \\
      \qquad \text{end} \\
      \text{end}
    \end{array}
  \]
\end{frame}

\begin{frame}{Contribution to open source project}
  \begin{block}{Alex: A lexical analyser generator for Haskell}
    \begin{itemize}
      \item
        Found some inefficiency in the alex's implementation of Hopcroft's algorithm
        \pause
      \item
        I improved it in various ways
        \begin{enumerate}
          \item Use better data structure and functions
          \alert<3>{\item Exclude trivial cases}
          \item The idea described before in this presentation
        \end{enumerate}
        \url{https://github.com/haskell/alex/pull/176} \\
        \pause
        \alert<3>{Most performance gain was came from 2.}
    \end{itemize}
  \end{block}
\end{frame}

\begin{frame}{References}
  Re-describing an algorithm by Hopcroft(Timo Knuutila, 2001)
\end{frame}

\end{document}
