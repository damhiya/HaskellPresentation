\documentclass{beamer}
\usetheme{metropolis}
\usecolortheme{seahorse}

\usepackage{kotex, minted, csquotes}

\usepackage{fontspec}
\setsansfont{Noto Sans}
\setsanshangulfont{Noto Sans CJK KR}
\setmonofont{Iosevka Fixed Slab}

\usepackage{unicode-math}
\setmathfont{STIX Two Math}

\newcommand\onlym[2]{\only<#1>{#2}}
\newcommand\alertm[2]{\alert<#1>{#2}}

\title{ST 모나드와 다형성}
\subtitle{rank-n type \& impredicative type}
\author{문순원}
\date{}

\begin{document}

\begin{frame}
  \titlepage
\end{frame}

\begin{frame}[fragile]
  \frametitle{ST 모나드의 필요}
  \begin{block}<1-> {가변자료}
    \begin{itemize}
      \item 많은 상황에서 효율성을 위해 가변자료형이 필요\\
      \item 하스켈에서는 가변상태 제어에 IO 모나드를 사용
    \end{itemize}
  \end{block}

  \begin{block}<2-> {IO 모나드의 문제점}
    \begin{itemize}
      \item IO 모나드는 가변자료 이상의 기능을 제공
      \item IO 모나드의 연산결과는 IO 타입을 탈출할수 없다
      \item IO 모나드를 강제로 탈출할경우 참조투명성이 훼손 \\
\begin{minted}{haskell}
unsafePerformIO :: IO a -> a
\end{minted}
    \end{itemize}
  \end{block}
\end{frame}

\begin{frame}{ST 모나드의 개선점}
  \begin{displayquote}
    The ST monad allows for destructive updates, but is escapable (unlike IO)
    \par\raggedleft--- \textup{\texttt{base}} 라이브러리 문서
  \end{displayquote}
  \pause
  \begin{itemize}
    \item ST 모나드는 결정론적이다
    \item ST 모나드는 탈출할수 있으면서도 참조투명성을 해치지 않는다
    \item \textbf{순수 함수 인터페이스를 가진 명령형 코드}
  \end{itemize}
\end{frame}

\begin{frame}[fragile] {어떻게 가능한가?}
\begin{minted}{haskell}
runST :: (forall s. ST s a) -> a
\end{minted}
  \pause
  \begin{itemize}
    \item ST 모나드를 안전하게 탈출
    \item \textbf{rank-2 다형성}
  \end{itemize}
\end{frame}

\begin{frame}[fragile] {다형함수를 인자로 받기}
\begin{minted}[escapeinside=||]{haskell}
foo :: |\onlym{1-2}{??}\onlym{3-}{(forall a. a -> a) -> (Int, Bool)}|
foo f = (f 1, f True)

bar :: (Int, Bool)
bar = foo |\onlym{1}{id}\onlym{2-}{(id :: forall a. a -> a)}|
\end{minted}
  \action<4-> {
    \begin{itemize}
      \item rank-2 타입
      \item \textbf{표준 하스켈에서는 허용되지 않음}
    \end{itemize}
  }
\end{frame}

\begin{frame} {rank에 따른 타입의 분류}
  \begin{displayquote}
  The \textbf{rank} of a type describes the depth at which universal quantifiers appear contravariantly
  \par\raggedleft---Practical type inference for arbitrary-rank types
  \end{displayquote}
  
  \pause
  \begin{block}{rank-0}
    \begin{itemize}\alert<6>{\item\texttt{Int -> Int}}\end{itemize}
  \end{block}

  \pause
  \begin{block}{rank-1}
    \begin{itemize}
      \alert<6>{\item\texttt{forall a. Int -> a -> a}}
      \item \texttt{Int -> (forall a. a -> a)}
    \end{itemize}
  \end{block}

  \pause
  \begin{block}{rank-2}
  \begin{itemize}\item\texttt{(forall a. a -> a) -> Int}\end{itemize}
  \end{block}

  \pause
  \begin{block}{rank-3}
    \begin{itemize}
      \item \texttt{((forall a. a -> a) -> Int) -> Int}
    \end{itemize}
  \end{block}
\end{frame}

\begin{frame} {rank에 따른 타입시스템의 분류}
  \begin{itemize}
    \alert<2>{
    \item rank-1 다형성 : rank가 1 이하인 타입을 허용
    \item rank-2 다형성 : rank가 2 이하인 타입을 허용
    }
    \item rank-3 다형성 : rank가 3 이하인 타입을 허용
      \\\texttt{...}
    \item rank-k 다형성 : rank가 k 이하인 타입을 허용
      \vfill
    \item rank-n (higher-rank) 다형성 : 임의의 rank를 가진 타입을 허용
  \end{itemize}
  \pause
  \alert<2>{타입추론이 결정가능}
\end{frame}

\begin{frame}[fragile] {GHC의 언어 확장 \texttt{RankNTypes}}
  \begin{block} {\texttt{RankNTypes}}
    \begin{itemize}
      \item 하스켈에 rank-n 다형성 지원을 추가\\
      \pause
      \item 이제는 가능하다
\begin{minted}{haskell}
foo :: (forall a. a -> a) -> (Int, Bool)
foo f = (f 1, f True)
\end{minted}
    \end{itemize}
  \end{block}
  \pause
  \begin{block} {\texttt{Rank2Types}}
    \begin{itemize}
      \item 이론적으로 타입추론이 결정가능
      \item 실제로 GHC에 결정가능한 rank-2 타입추론이 구현된적은 없음
      \pause
      \item \texttt{RankNTypes}의 별칭
      \item 쓰지마세요
    \end{itemize}
  \end{block}
\end{frame}

\begin{frame}[fragile] {rank vs kind}
  \begin{itemize}
    \item rank \(\neq\) kind
      \pause
\begin{minted}[escapeinside=||]{text}
Int -> Int                   :: Type |\textcolor{gray}{\textit{(rank-0)}}|
forall a. Int -> a -> a      :: Type |\textcolor{gray}{\textit{(rank-1)}}|
Int -> (forall a. a -> a)    :: Type |\textcolor{gray}{\textit{(rank-1)}}|
(forall a. a -> a) -> Int    :: Type |\textcolor{gray}{\textit{(rank-2)}}|
\end{minted}
    \pause
  \item rank에 상관없이 kind \texttt{\enquote*{Type}}을 가진다
  \pause
  \item 하스켈에서는 rank에 따른 차별이 존재
\begin{minted}{haskell}
[not] :: [Bool -> Bool]     -- okay
[id]  :: [forall a. a -> a] -- not okay
\end{minted}
  \end{itemize}
  \pause
  \begin{displayquote}
    polymorphic types are themselves not first class
    \par\raggedleft---A Quick Look at Impredicativity
  \end{displayquote}
\end{frame}

\begin{frame}[fragile] {Impredicativity}
\begin{minted}{haskell}
id :: forall p. p -> p
mono :: [Bool] -> [Bool]
poly :: forall a. [a] -> [a]
\end{minted}
  \text{\newline}
  \begin{overprint}
    \onslide<1>
\begin{minted}{haskell}
mono_a :: [Bool] -> [Bool]
mono_a = id @([Bool] -> [Bool]) mono

poly_a :: forall b. [b] -> [b]
poly_a @b = id @([b] -> [b]) (poly @b)

poly_b :: forall b. [b] -> [b]
poly_b = id @(forall a. [a] -> [a]) poly
\end{minted}
    \onslide<2>
\begin{minted}{haskell}
mono_a :: [Bool] -> [Bool]
mono_a = id @([Bool] -> [Bool]) mono

p ~ [Bool] -> [Bool]
\end{minted}
    \onslide<3>
\begin{minted}{haskell}
poly_a :: forall b. [b] -> [b]
poly_a @b = id @([b] -> [b]) (poly @b)

p ~ [b] -> [b]
a ~ b
\end{minted}
    \onslide<4-5>
\begin{minted}[escapeinside=||]{haskell}
poly_b :: forall b. [b] -> [b]
poly_b = id @(forall a. [a] -> [a]) poly

|\alertm{5}{p ~ forall a. [a] -> [a]}|
\end{minted}
    \only<5-> {
      \begin{itemize}
        \alert<5> {\item 타입변수를 다형타입으로 인스턴스화}
      \end{itemize}
    }
    \onslide<6->
\begin{minted}[escapeinside=||]{haskell}
poly_b :: forall b. [b] -> [b]
poly_b = id @(forall a. [a] -> [a]) poly

p ~ forall a. [a] -> [a]
\end{minted}
    \begin{itemize}
      \item 타입변수를 다형타입으로 인스턴스화
      \item Impredicative 다형성
      \item rank-n 다형성의 일반화\\
        \texttt{[id] :: [forall a. a -> a]}
      \item GHC는 `아직' impredicativity를 제대로 지원하지 않음
    \end{itemize}
  \end{overprint}
\end{frame}

\begin{frame}[fragile] {\texttt{ST} \& \texttt{runST}}
\begin{minted}{haskell}
action :: ST s a
runST :: forall a. (forall s. ST s a) -> a
\end{minted}
  \vfill
  \begin{displayquote}
    A computation of type \texttt{ST s a} returns a value of type \texttt{a}, and execute in ``thread'' \texttt{s}.\\
    \texttt{runST} return the value computed by a state thread. The forall ensures that the internal state used by the ST computation is inaccessible to the rest of the program.
    \par\raggedleft--- \textup{\texttt{base}} 라이브러리 문서
  \end{displayquote}
  \pause
  \begin{itemize}
    \item ST 모나드의 연산결과를 ST 모나드에서 탈출시킴
    \item 내부 상태가 ST 모나드를 탈출하지 못하게 함
  \end{itemize}
\end{frame}

\begin{frame}[fragile] {ST 모나드의 내부 상태}
  \begin{block}{내부 상태 \texttt{≈} 참조 타입}
    \begin{itemize}
      \item \texttt{STRef s a}
      \item \texttt{MVector s a}
      \item \texttt{HashTable s k a}
      \pause
      \item 스레드 \texttt{s}에 종속된 \texttt{a} 타입의 가변 상태를 가리키는 참조
      \item \texttt{ST} 모나드 안에서 값을 읽고 쓸수 있음
\begin{minted}{haskell}
newSTRef :: a -> ST s (STRef s a)
readSTRef :: STRef s a -> ST s a
writeSTRef :: STRef s a -> a -> ST s ()
\end{minted}
    \end{itemize}
  \end{block}
\end{frame}

\begin{frame}[fragile] {\texttt{runST}의 참조투명성 보장}
  \begin{overprint}
    \onslide<1>
\begin{minted}{haskell}
runST :: forall a. (forall s. ST s a) -> a

escapeInt :: Int
escapeInt = runST action where
  action :: forall s. ST s Int
  action = do
    ref <- newSTRef 0
    readSTRef ref
\end{minted}
    \texttt{\newline forall s. ST s a \textasciitilde \ forall s. ST s Int}\\
    \(\therefore\) \texttt{a \textasciitilde \ Int}
    \begin{itemize}
      \item \texttt{Int}는 ST 모나드를 탈출할 수 있음
    \end{itemize}
    \onslide<2>
\begin{minted}{haskell}
runST :: forall a. (forall s. ST s a) -> a

escapeSTRef :: STRef s Int..?
escapeSTRef = runST action where
  action :: forall s. ST s (STRef s Int)
  action = do
    ref <- newSTRef 0
    return ref
\end{minted}
    \texttt{\newline forall s. ST s a \textasciitilde \ forall s. ST s (STRef s Int)}\\
    \(\therefore\) \texttt{a \textasciitilde \ STRef s Int}\\
    \textbf{Couldn't match type \enquote*{\texttt{a}} with \enquote*{\texttt{STRef s Bool}} because type variable \enquote*{\texttt{s}} would escape its scope}
  \begin{itemize}
    \item 내부상태 \texttt{STRef s Int}는 ST 모나드를 탈출할 수 없음
  \end{itemize}
  \end{overprint}
\end{frame}

\begin{frame}[fragile] {ST 모나드와 벡터}
\begin{minted}[fontsize=\small]{haskell}
freeze       :: MVector s a -> ST s (Vector a)
unsafeFreeze :: MVector s a -> ST s (Vector a) 

thaw       :: Vector a -> ST s (MVector s a)
unsafeThaw :: Vector a -> ST s (MVector s a)

runST  :: (forall s. ST s a) -> a
create :: (forall s. ST s (MVector s a)) -> Vector a
\end{minted}
  \begin{itemize}
    \item 불변벡터-가변벡터간 변환
    \pause
    \item 복사를 하지 않는 구현은 참조 투명성을 깨뜨림
    \item \texttt{create}는 안전하면서도 복사가 없음
  \end{itemize}
\end{frame}

\begin{frame}[fragile] {\texttt{create} 사용 예시 (병합 정렬)}
  \text{\footnotesize \url{https://gist.github.com/damhiya/d46e7197b5186795f2fd3ae49eade029}}
  \begin{overprint}
    \onslide<1>
\begin{minted}[escapeinside=||] {haskell}
sortM :: Ord a => MVector s a -> ST s ()
sortM u = ..

sort :: Ord a => Vector a -> Vector a
sort v =
  if V.length v <= 1 then
    v
  else runST $ do
    v' <- V.thaw v
    sortM v'
    V.freeze v'
\end{minted}
    \onslide<2>
\begin{minted}[escapeinside=||] {haskell}
sortM :: Ord a => MVector s a -> ST s ()
sortM u = ..

sort :: Ord a => Vector a -> Vector a
sort v =
  if V.length v <= 1 then
    v
  else runST $ do
    v' <- V.thaw v
    sortM v'
    V.unsafeFreeze v'
\end{minted}

    \onslide<3>
\begin{minted}[escapeinside=||] {haskell}
sortM :: Ord a => MVector s a -> ST s ()
sortM u = ..

sort :: Ord a => Vector a -> Vector a
sort v =
  if V.length v <= 1 then
    v
  else V.create $ do
    v' <- V.thaw v
    sortM v'
    return v'
\end{minted}
      \begin{itemize}
        \item 불필요한 복사 제거
        \item 참조 투명성이 항상 보장됨
      \end{itemize}

  \end{overprint}
\end{frame}

\begin{frame} {Q\&A}
\end{frame}
\end{document}
