\documentclass[10pt]{beamer}
\usefonttheme[onlymath]{serif}

\usepackage{kotex, csquotes, qtree, stmaryrd, tikz-cd, proof}

% Mathematical notations
\newcommand{\pow}[0]{\mathcal{P}}
\newcommand{\chr}[1]{\textrm{`\(#1\)'}}

% Imp lang
\newcommand{\Id}[0]{\mathrm{Id}}
\newcommand{\AExp}[0]{\mathrm{AExp}}
\newcommand{\BExp}[0]{\mathrm{BExp}}
\newcommand{\Stmt}[0]{\mathrm{Stmt}}
\newcommand{\true}[0]{\texttt{true}}
\newcommand{\false}[0]{\texttt{false}}
\newcommand{\asn}[0]{\leftarrow}
\newcommand{\skp}[0]{\texttt{skip}}
\newcommand{\ifte}[3]{\texttt{if} \ #1 \ \texttt{then} \ #2 \ \texttt{else} \ #3}
\newcommand{\while}[2]{\texttt{while} \ #1 \ \texttt{do} \ #2 }
\newcommand{\putn}[1]{\texttt{put} \ #1}
\newcommand{\getn}[0]{\texttt{get}}

% Eval
\newcommand{\Mem}[0]{\mathrm{Mem}}
\newcommand{\True}[0]{\mathrm{T}}
\newcommand{\False}[0]{\mathrm{F}}
\newcommand{\eval}[0]{\mathrm{eval}}

% Semantics
\newcommand{\bigstep}[3]{#1 \vdash #2 \Downarrow #3}
\newcommand{\Trace}[0]{\mathrm{Trace}}
\newcommand{\beh}[0]{\mathrm{beh}}

% Hoare logic
\newcommand{\hoare}[3]{\{#1\} #2 \{#3\}}

\title{프로그래밍 언어의 의미}
\author{문순원}

\begin{document}

\begin{frame}
  \titlepage
\end{frame}

\begin{frame}{소프트웨어 안정성을 높이는 방법}
  \begin{block}{테스팅}
    \begin{itemize}
      \item 몇 가지 케이스에서 프로그램의 실행 결과를 기대되는 결과와 비교
      \item 테스트를 통과했더라도 버그가 없음을 확신할 수는 없음
    \end{itemize}
  \end{block}

 \pause
 \begin{block}{프로그램 분석}
   \begin{itemize}
     \item 프로그램을 자동으로 분석하는 프로그램
     \item 정지문제에 따른 한계
   \end{itemize}
 \end{block}

 \pause
  \begin{block}{모델 체킹}
    \begin{itemize}
      \item 상태 전이 기계의 특정 성질을 자동으로 검증
      \item State explosion problem을 피하기 쉽지 않음
    \end{itemize}
  \end{block}

 \pause
  \begin{block}{정형 검증}
    \begin{itemize}
      \item 사람이 직접 프로그램의 성질을 증명
      \item 굉장히 높은 비용
    \end{itemize}
  \end{block}
\end{frame}

\begin{frame}{정형 검증}
  정형 검증 (Formal verification)
  \begin{displayquote}
  the act of proving or disproving the correctness of intended algorithms underlying a system with respect to a certain formal specification or property, using formal methods of mathematics
  --- from Wikipedia
  \end{displayquote}
  \pause
  \begin{itemize}
    \item 프로그램과 그 스펙을 수학적 대상으로 다룸
    \item 연역적 방법을 사용해 프로그램의 성질을 증명
    \item 증명의 무결성을 컴퓨터로 검사할 수 있음
  \end{itemize}
\end{frame}

\begin{frame}{프로그램 정의의 계층}
  \begin{itemize}
    \item 기호들의 나열 (concrete syntax) \qquad \quad \
      \( \left[ \chr{2}, \chr{+}, \chr{3}, \chr{\times}, \chr{4} \right] \)
    \item<2-> 추상 구문 트리 (abtract syntax)
      \Tree [.\(+\) \(2\) [.\(\times\) \(3\) \(4\) ] ] \\
    \item<3-> 의미 (semantics) \\
      프로그래밍 언어의 패러다임, 의미의 용도에 따라 여러가지 접근이 있다.
      \onslide<4->
      \begin{itemize}
        \item Operational semantics
          \begin{itemize}
            \item Big-step semantics
            \item Small-step semantics
          \end{itemize}
        \item Denotational semantics
        \item Axiomatic semantics
      \end{itemize}
  \end{itemize}
\end{frame}

\begin{frame}{Imp}
  \only<1-2> {
    간단한 프로그래밍 언어 Imp를 통해 추상 문법과 의미를 정의하는 방법을 알아보자.

    \onslide<2->
    \[
      \begin{array}{llrl}
        \Id   & x, y, z & & \\
        \AExp & E & ::= & n
                     \mid x
                     \mid E_1 + E_2
                     \mid - E
                     \mid E_1 \times E_2
                     \mid \ldots
                     \\
        \BExp & B & ::= & \true
                     \mid \false
                     \mid \neg B
                     \mid B_1 \wedge B_2
                     \mid B_1 \vee B_2
                     \mid \\
                    & & & E_1 = E_2
                     \mid E_1 \leq E_2
                     \mid \ldots
                     \\
        \Stmt & S & ::= & \skp
                     \mid x \asn E
                     \mid S_1 ; S_2
                     \mid \ifte{B}{S_1}{S_2}
                     \mid \while{B}{S}
      \end{array}
    \]
    \begin{itemize}
      \item 정수와 논리 연산이 가능한 명령형 언어
      \item 변수는 정수만을 담을 수 있다
      \item 제어 구문은 if, while 뿐
      \item \(\AExp\), \(\BExp\), \(\Stmt\)는 귀납적으로 정의된 추상 문법 트리들의 집합
    \end{itemize}
  }

  \only<3-4> {
    10이하 자연수들의 합을 구하는 프로그램
    \[ x \asn 10 ; y \asn 0 ; \while{0 \leq x}{ \{ y \asn y + x ; x \asn x - 1 \} } \]

    \onslide<4->
    \Tree [.\(;\) [.\(\asn\) \(x\) \(10\) ]
                  [.\(;\) [.\(\asn\) \(y\) \(0\) ]
                          [.\(\while{..}{..}\) [.\(\leq\) \(0\) \(x\) ]
                                               [.\(;\) [.\(\asn\) \(y\)
                                                                        [.\(+\) \(y\) \(x\) ] ]
                                                       [.\(\asn\) \(x\)
                                                                        [.\(-\) \(x\) \(1\) ] ] ] ] ] ]
  }
\end{frame}

\begin{frame}{메모리 상태}
  \only<1> {
    명령형 프로그래밍 언어는 변수의 상태를 바꿈으로서 계산을 수행하기 때문에,
    이러한 언어의 의미를 다루기 위해서는 먼저 메모리를 정의할 필요가 있다.
    Imp의 메모리 상태는 다음처럼 정의할 수 있다.
    \[ M \in \Mem = \mathbb{Z}^{\Id} \]
  }

  \only<2> {
    메모리의 상태가 정해져있다면 \(\AExp\)와 \(\BExp\)의 값을 구할 수 있다.
    \[
      \begin{array}{l}
        \eval : \Mem \times \AExp \to \mathbb{Z} \\
        \eval : \Mem \times \BExp \to \mathbb{B} \\
      \end{array}
    \]
    \[
      \begin{array}{rcl}
        \eval(M,n) &=& n \\
        \eval(M,x) &=& M(x) \\
        \eval(M,E_1 + E_2) &=& \eval(M,E_1) + \eval(M,E_2) \\
        & ... & \\
        \eval(M,\true) &=& \True \\
        \eval(M,\false) &=& \False \\
        \eval(M,B_1 \wedge B_2) &=& \eval(M,B_1) \wedge \eval(M,B_2) \\
        \eval(M,E_1 = E_2) &=&
        \begin{cases}
          \True, & \text{if}\ \eval(M,E_1) = \eval(M,E_2) \\
          \False, & \text{otherwise}
        \end{cases}
        \\
        & ... &
      \end{array}
    \]
  }
\end{frame}

\begin{frame}{Imp의 의미 - Big-step semantics}
  \only<1-3> {
    명령형 언어의 의미를 메모리의 상태를 바꾸는 함수로 이해할 수 있지 않을까?
    \[ \Stmt \times \Mem \to \Mem \]

    \onslide<2->
    \begin{itemize}
      \item 프로그램이 종료하지 않는다면?
      \item 프로그램이 비결정적이라면?
    \end{itemize}

    \onslide<3->
    명령형 언어의 의미를 메모리의 상태를 바꾸는 `관계'로 정의하자.
    \[ \Stmt \to \pow(\Mem \times \Mem) \]
    \[ \bigstep{M_1}{S}{M_2} \]
    ``상태 \(M_1\)에서 프로그램 \(S\)가 실행되었을 때 상태 \(M_2\)로 종료할 수 있다''
  }

  \only<4> {
    \small

    \infer{\bigstep{M}{\skp}{M}}{
    }\vspace{1em}

    \infer{\bigstep{M}{x \asn E}{M [x \mapsto v]}}{
      \eval(M,E) = v
    }\vspace{1em}

    \infer{\bigstep{M_1}{S_1;S_2}{M_3}}{
      \bigstep{M_1}{S_1}{M_2} &
      \bigstep{M_2}{S_2}{M_3}
    }\vspace{1em}

    \infer{\bigstep{M_1}{\ifte{B}{S_1}{S_2}}{M_2}}{
      \eval(M_1,B) = \True &
      \bigstep{M_1}{S_1}{M_2}
    }\vspace{1em}

    \infer{\bigstep{M_1}{\ifte{B}{S_1}{S_2}}{M_2}}{
      \eval(M,B) = \False &
      \bigstep{M_1}{S_2}{M_2}
    }\vspace{1em}

    \infer{\bigstep{M_1}{\while{B}{S}}{M_3}}{
      \eval(M,B) = \True &
      \bigstep{M_1}{S}{M_2} &
      \bigstep{M_2}{\while{B}{S}}{M_3}
    }\vspace{1em}

    \infer{\bigstep{M}{\while{B}{S}}{M}}{
      \eval(M,B) = \False
    }
  }

  \only<5> {
    상태 \(M_1\)에서 프로그램 \(S\)가 무한루프를 도는 경우에는
    \[ \{ M_2 \mid \bigstep{M_1}{S}{M_2} \} = \emptyset \]
    현실에는 서버와 같이 무한루프를 도는 유용한 프로그램도 존재한다.
  }
\end{frame}

\begin{frame}{Imp의 의미 - Small-step semantics}
  \only<1-2> {
    \begin{block}{전이 시스템 (transition system)}
      상태들의 집합 \(S\), 전이 관계 \(\to \ \subseteq S \times S\). \\
      \( (p,q) \in \ \to \)는 \( p \to q \)와 같이 표기한다. \newline

      \onslide<2->
      Imp의 의미를 \(S = (\Stmt \uplus \{\cdot\}) \times \Mem\)을 상태로 가지는 전이 시스템으로 정의할 수 있다.
      \[ \pow(S \times S) \]

      \(\Stmt \uplus \{\cdot\}\)는 프로그램 카운터와 비슷한 역할
    \end{block}
  }

  \only<3-4> {
    \[ S = x \asn 1; y \asn 2; z \asn x + y \]

    \onslide<4->
    \(\{ \ast \mapsto 0\}\)에서 \(S\)를 실행시키는 경우
    \[
      \begin{tikzcd}
        (x \asn 1; y \asn 2; z \asn x + y, \{\ast \mapsto 0\}) \\
        (y \asn 2; z \asn x + y, \{x \mapsto 1, \ast \mapsto 0\}) \\
        (z \asn x + y, \{y \mapsto 2, x \mapsto 1, \ast \mapsto 0\}) \\
        (\cdot, \{z \mapsto 3, y \mapsto 2, x \mapsto 1, \ast \mapsto 0\}) \\
	\arrow[from=1-1, to=2-1]
	\arrow[from=2-1, to=3-1]
	\arrow[from=3-1, to=4-1]
      \end{tikzcd}
    \]
  }

  \only<5-6> {
    현실의 프로그램들은 외부 세계와 상호작용(키보드 입력, 화면 출력, 디스크 작업, ...)을 하는데, 이것을 어떻게 표현할 수 있을까? \newline

    \onslide<6->
    Imp를 문법을 다음과 같이 확장시켜보자.
    \[
      \begin{array}{llrl}
        \Stmt & S & ::= & \skp
                     \mid x \asn E
                     \mid S_1 ; S_2
                     \mid \ifte{B}{S_1}{S_2}
                     \mid \while{B}{S}
                     \mid \\
                    & & & {\color{blue} \putn{E} }
                     \mid {\color{blue} x \asn \getn }
      \end{array}
    \]
  }

  \only<7-9> {
    \begin{block}{라벨이 있는 전이 시스템 (labeled transition system)}
      상태들의 집합 \(S\), 라벨들의 집합 \(L\), 전이 관계 \(\to \ \subseteq S \times L \times S\)
      \[\def\arraystretch{1.5}
        \begin{array}{c}
          p \stackrel{e}{\to} q \qquad \text{where \(p, q \in S\) and \(e \in L\)} \\
          \text{``상태 \(p\)에서 이벤트 \(e\)를 발생시키며 상태 \(q\)로 전이할 수 있다''}
        \end{array}
      \]

      \onslide<8->
      \[ \pow(S \times L \times S) \]
      \[ S = (\Stmt \uplus \{\cdot\}) \times \Mem \]
      \[ L = \{ \putn{n} \mid n \in \mathbb{Z} \} \cup \{ \getn \ n \mid n \in \mathbb{Z} \} \]

      \onslide<9->
      이벤트는 프로그램과 외부세계의 상호작용을 표현한다.
      외부에서 관측 되는것이 이벤트 뿐이라면, 상태를 배제하고 이벤트만으로 프로그램의 의미를 정의할 수 있지 않을까?
    \end{block}
  }
\end{frame}

\begin{frame}{Imp의 의미 - 트레이스, 행동}
  \begin{block}{트레이스 (trace)}
    프로그램의 한 실행에서 발생한 관측 가능한 이벤트들의 시퀸스. \\
    프로그램의 행동 \(\beh\)는 발생 가능한 모든 트레이스들의 집합.
    \[ \beh : \Stmt \to \pow(\Trace(L)) \]

    \pause
    \(\Trace\)는 종료하지 않는 프로그램의 트레이스를 표현할 수 있도록 주의 깊게 정의 되어야 한다.
    CompCert에서는 프로그램의 트레이스를 다음과 같이 4가지로 분류한다.
    { \footnotesize
      \begin{itemize}
        \item Termination, 유한개의 관측 가능한 이벤트를 발생시킨 후 종료한 경우.
        \item Divergence, 유한개의 관측 가능한 이벤트를 발생시킨 후 무한히 실행되며 종료하지 않는 경우.
        \item Reactive divergence, 무한개의 관측 가능한 이벤트를 지속적으로 발생시키는 경우.
        \item Going wrong, 유한개의 관측 가능한 이벤트를 발생시킨 후 비정상적으로 종료한 경우.
      \end{itemize}
    }
  \end{block}
\end{frame}

% \begin{frame}{프로그램의 스펙}
%   \only<1-> {
%     \begin{block}{호어 트리플}
%       \(\hoare{P}{C}{Q}\) \ : 선행조건 \(P\)를 만족하는 상태에서 프로그램 \(C\)를 실행하여 \textbf{종료한 경우}, 종료한 후의 상태는 항상 후행조건 \(Q\)를 만족한다.
%     \end{block}
%   }
% \end{frame}

\end{document}
